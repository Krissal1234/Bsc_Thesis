

% in methodology - Teh approach

%in the intro:
% motivation
% hyp and resea
% aims and objec
% socument structure


\section*{Introduction}
\subsection*{Introduction to Skateboarding and its Evolution}
\paragraph{}
Skateboarding dates back to the 1940s when handmade skateboards first appeared
\cite{SkateboardingEncyclobeadia}. It has since developed into a worldwide
phenomenon, with its popularity skyrocketing, after gaining recognition as an
official sport in the 2020 Tokyo Olympic Games \cite{SkatebaordingOlimpics}.
Skateboarding comprises the dynamic activities of riding a skateboard and
skillfully performing a repertoire of tricks, manifesting as a popular and
exhilarating “extreme sport”. 
\paragraph{}
This dynamic sport encompasses various disciplines and riding styles, each
offering unique challenges for skateboarders to explore. Two of the most
prominent styles are “vert” and “street.” Vert skateboarding revolves around
riding on specialised obstacles, namely, half-pipes and ramps, emphasising
aerial manoeuvres. Street skateboarding transpires in urban environments,
utilising various obstacles that can be found outdoors, including stairs, rails,
ledges, gaps or flat ground for skaters to showcase their creativity
\cite{skateStyles}.


\subsection*{Motivation}
\paragraph{}
The current method of identifying tricks in skateboard competitions relies
heavily on judges verbally stating them during live broadcasts. This approach
lacks a digital overlay that could display the performed trick for the viewers,
leading to a reliance on subjective judgment. Such subjectivity can lead to
scoring disparities and conflicts, undermining competition fairness.
Furthermore, this absence of an objective identification method obstructs the
skaters' ability to receive real-time feedback, which could be beneficial for
skill development. An AI-based system that delivers a dependable and efficient
method of trick recognition could improve this aspect and contribute to the lack
of study in this area. 

\section*{Scope}
\paragraph{}
The scope of this study is not only to develop a skateboard trick classifier but
also to delve into the largely unexplored field of Artificial Intelligence (AI)
in action sports. Studying AI in this domain opens up possibilities for
innovation in the skateboarding industry, such as introducing fairer scoring
systems in competitions and enhancing the live-streaming viewing experience.
\paragraph{}
Moreover, this study offers various research opportunities, like providing
detailed coaching feedback, where, in the case of skateboarding, Machine
Learning (ML) could detect inaccuracies in body weight placement and foot
positioning. Additionally, this study opens avenues for injury prevention
research by identifying common injury patterns and risky manoeuvres. 



\section*{Brief overview of the literature} 
\paragraph{}
Recent advancements in recognising human actions in videos have significantly
impacted various fields, ranging from the medical sector to surveillance
systems, \cite{3DhumanActionDetectionForHealthCareSystems},
\cite{HumanActivityRecognitionSecurityAndMonitoring}. Particularly noteworthy is
its application in developing an AI-based skateboard trick classifier, an area
that has seen limited research.

\paragraph{}
In this emergent field, leveraging activity recognition techniques from a video
has led to two primary methodologies among researchers. The first technique
involves utilising signals obtained from skateboard-mounted accelerometers or
signals artificially generated based on the findings of prior studies. These
signals are then fed into a study-dependent model for classification, as
outlined in \cite{skateboardClassificationTransferLearningPipelinesAccelermetry}
and \cite{skateboardTrickClassifierUsingAccelerometryAndML}. The second approach
employs computer vision techniques, leveraging video footage of skateboard
tricks to train and refine models for accurate trick identification, as depicted
by the studies \cite{skatePaper1} and \cite{SkateboardAIPaper}. 


\subsection*{Accelerometry approach}
\paragraph{}
The study by Abdullah et al (2021)
\cite{skateboardClassificationTransferLearningPipelinesAccelermetry}, makes use
of a custom dataset comprising of six skateboard tricks most commonly executed
in competitive events. Amateur skateboarders performed each trick five times on
a modified skateboard equipped with an Inertial Measurement Unit (IMU) to record
the signals produced. The researchers capture six signals for each trick,
including linear accelerations along the x, y, and z axes (aX, aY, aZ) and
angular accelerations along the same axes (gX, gY, gZ). They then opt for the
unique approach of concatenating all six signals onto a single image
corresponding to one trick, employing two input image transformations: raw data
(RAW) and Continuous Wavelet Transform (CWT). With the application of six
transfer learning models on this data, the results show that RAW and CWT input
images on MobileNet, MobileNetV2 and ResNet101 models achieved favourable
accuracy. However, the CWT-MobileNet-Optimised SVM pipeline was deemed the best
due to its reduction in computational time.

\paragraph{}
The study by Corrêa et al (2017)
\cite{skateboardTrickClassifierUsingAccelerometryAndML}, obtained their sample
data by artificially generating 543 signals based on prior research, utilising
tools such as MATLAB 2015 and Signal Processing Toolbox. These signals were then
categorised into five distinct classes representing different skateboard tricks,
each with various samples ranging from 30 to 50 per class, across three axes (X,
Y and Z). This study developed and validated individual Artificial Neural
Networks (ANNs) for each axis, as well as the combination of the three: ANN XYZ,
displaying the potential of Neural Networks to categorise multidimensional
skateboard tricks. The ANNs are all multilayer feed-forward neural networks
(MFFNNs), structured into three distinct layers. They feature an input layer
with 82 neurons, a hidden layer, comprised of 23 neurons utilising a tan-sigmoid
transfer function and an output layer consisting of 5 neurons with a softmax
function. Finally, the study achieved high accuracies, with ANNs X, Y and Z
achieving accuracies of 94.8\%, 96.7\% and 98.7\%, respectively, while the
combined ANN XYZ achieved an accuracy of 92.8\%. 

\subsection*{Computer Vision approach}
\paragraph{}
The paper by Shapiee et al (2020) \cite{skatePaper1} leverages a custom data set
comprising videos capturing the execution of five distinct skateboard tricks,
each attempted five times. Each video spans two to three seconds, yielding a
total of 750 images by extracting 30 frames per video. This study made use of
data augmentation techniques to expand their dataset further. Consequently, they
introduced an additional 2,250 images, achieving 3,000 images in their data set.
On the other hand, Chen (2023) \cite{SkateboardAIPaper} compiled a comprehensive
data set by collecting videos from multiple platforms, including YouTube,
Twitter and Instagram. Furthermore, Chen trained the model using 15 fundamental
tricks commonly observed in competitive settings. The researcher collected 50
videos per trick, summing up to a total number of 750 videos. Of these, 45
videos per trick were allocated for training, and the remaining 5 were reserved
for validation. 

\paragraph{}
Data augmentation techniques are popular in studies with relatively limited
datasets. Techniques such as flipping, rotating, scaling and colour manipulation
not only enhance the size of the original dataset but also lower the likelihood
of the model overfitting \cite{DataAugmentationCanImproveRobustness}. The paper
by Shapiee et al \cite{skatePaper1} utilises three rotation augmentation
techniques: horizontal rotation, positive 90°rotation and negative 90°rotation.
The researchers experimented on three Transfer learning models: MobileNet,
NASNetMobile and NASNetLarge, each evaluated using a k-Nearest Neighbor (k-NN)
classifier. As a result, the models demonstrated impressive classification
accuracies, with MobileNet achieving 95\%, NASNetMobile 92\% and NASNetLarge
90\%. 

\paragraph{}
Recurrent Neural Networks (RNNs) and Long Short-Term Memory Networks (LSTMs) are
popular architectures due to their capabilities in modelling the dynamic
relationships in sequential data \cite{UnderstandingLSTM}. In the student
abstract by Hanciao Chen \cite{SkateboardAIPaper}, extensive experimentation is
conducted using diverse models, exploring various combinations of CNN-LSTM and
CNN-BiLSTM, including integrating attention and transfer learning approaches.
This study further documents and analyses important metrics such as training
time, training accuracy and validation accuracy for each model experimented on.
Among these, the top three models that stood out in terms of validation accuracy
were the ResNet50 with Attention and BiLSTM (84\%), ResNet50 with BiLSTM (81\%)
and ResNet50 with LSTM (80\%). Chen's study provides valuable insight into the
application of diverse models in activity recognition in skateboarding. 



\section*{Aims and Objectives}
\paragraph{}
The study primarily aims to investigate the effectiveness of various deep
learning techniques to develop an Artificial Intelligence (AI) model that can
identify skateboard tricks directly from videos. This aim can be
satisfied by accomplishing the objectives below.
\subsubsection*{Research Question}
How effectively can a combination of deep learning techniques and algorithms
accurately classify skateboard tricks from video data?
\subsubsection*{Hypothesis}
Employing a combination of deep learning strategies and preprocessing techniques
can improve the accuracy and robustness in classifying skateboard tricks.
\subsubsection*{Objectives}
\normalsize \begin{itemize}
    \item \textbf{Assemble a comprehensive dataset:} Collect a diverse range of
    skateboard tricks performed in various styles and environments. This data will
    serve as the basis for hypothesis testing.
    \item \textbf{Investigate preprocessing techniques:} Investigate various
    preprocessing techniques and their impact on model outcomes.
    \item \textbf{Develop and test deep learning models:} Create and assess
    several models, emphasizing the investigation of architectures previously
    established in literature. These models will be used to assess the
    hypothesis's validity.
    \item \textbf{Analyse the model's performance:} Evaluate and compare the accuracy
    and efficiency of these models in the classification of skateboard tricks to
    validate or refute the hypothesis. 
    \item \textbf{Evaluate the hypothesis:} Make a final determination on
    whether the model's performance supports or contradicts the hypothesis.
\end{itemize}

% in classifying skateboard tricks. This study

% The aims and objectives of this project are to design and develop a dependable
% Artificial Intelligence (AI) model that can identify various skateboard tricks
% directly from videos. An initial focus will be placed on the recognition of three
% distinct tricks namely, kickflips, ollies and shuvits. To reach this goal the
% appropriate ML techniques must be utilised to overcome basic computer vision
% challenges, such as varying camera angles, lighting conditions and the
% fast-paced nature of skateboard tricks. By addressing these challenges, this
% project aims to establish a usable tool that can be used by skateboarders, coaches, and
% skateboard competitions.


\section*{List of deliverables}
\begin{itemize}
    
    \item Extensive dataset compilation
    \item Trained model with detailed accuracy metrics
    \item Documentation of experiments and their visualisations
\end{itemize}


\section*{Methodology and Evaluation Strategy}
\paragraph{}
In terms of data collection, this project will utilise a combination of the
open-source datasets provided by Fitzgerald et al (2020)
\cite{lightningdrop2020skateboardml} and Chen \cite{SkateboardAIPaper}.
Moreover, to ensure an unbiased and rigorous evaluation of the model's
performance, this study will incorporate a dataset, consisting of skateboard
tricks performed and recorded specifically for this study (by myself). 

\paragraph{}
During my thesis research, I initially began by experimenting with various
preprocessing techniques, such as applying frame extraction to videos and
employing object detection to selectively crop each frame, focusing on
encapsulating only the skater. This approach was designed to remove any
unnecessary overhead in each frame that might otherwise impede the learning
process. 

\paragraph{}
I commenced the development phase using Python, opting for YOLO (You Only Look
Once) as the object detection framework. The YOLO training process involved
gathering a diverse dataset and labelling each image featuring a skateboarder by
drawing a bounding box around each subject. The trained object detection model
takes an image as an input and returns the coordinates of the bounding box
encasing the skater's position, used to crop the image accordingly. 

\paragraph{}
After completing the experimentation phase with various preprocessing techniques
and creating a temporary preprocessing mechanism, I developed a baseline model.
A baseline model establishes a benchmark for all future modifications and
enhancements to it. This foundational model, characterised by its ConvLSTM
architecture, is designed explicitly for sequential data processing. It begins
with a ConvLSTM2D input layer, which features a 3x3 kernel size and employs a
'tanh' activation function. Subsequently, this model incorporates a MaxPooling3D
layer with a pool size 1x2x2, complemented by TimeDistributed Dropout layers to
prevent overfitting. The architecture consistently follows this pattern,
progressively increasing the number of filters, finally ending with a Dense
layer. After training, the baseline model achieved a favourable but improvable
validation accuracy of 72\%. 
\subsection*{Work Plan}
\paragraph{}
I plan to refine the baseline model further and aim to achieve a more efficient
and accurate model. The approach will involve applying various preprocessing and
training techniques, including but not limited to data augmentation and the
'leave-one-out' approach. Additionally, I intend to experiment with transfer
learning methods, incorporating models like ResNet or VGG for feature extraction
and exploring adjustments in architectures and hyperparameters. 


\subsection*{Project timeline}
\begin{figure}[h]
    \includegraphics[width=1\textwidth]{content/chapters/1_introduction/figures/newGantt.png}
    \caption{Estimated project timeline.}
    \label{fig:my_label}
\end{figure}