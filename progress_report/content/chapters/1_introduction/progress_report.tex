% \chapter{Progress Report}%
% \label{chp:Progress Report}
% include in intro:

% - show the importance of the topic by discussing the underutilisation of AI in sports
% - references to current research - mention some papers that have created skateboard ai's

% - explain the limits of the ai - developing an ai that can recognise atelast two different tricks effectively

% - how the paper will be organised
%  - e.g. An assessment will first be made of the performance of the privatized industries themselves, on an individual basis, and then the performance of the economy as a whole will be examined.
 
%  - thesis statement - this is that sentence or two that contain the focus of your essay and tell your reader what the essay’s aim is.
 
%  - CARS model
%  	MOVE 1: Establishing a Territory
% 	MOVE 2: Establishing a Niche 
% 	MOVE 3: Occupying the Niche
 
 
%  PLAN
 
%  - Definitions
 
%  - references to past research
 
 
%  - emphasizing hte importance of the topic
 
%  - I will establish the niche using the underutilisation of ai in sports and how under researched ai in skateboarding is
 
 
%  - show importnace of the topic
%  - explain limits of the research
%  - how the paper will be organised
 
%  - Thesis statement
 
\section*{Introduction}
Skateboarding dates back to the 1940s when handmade skateboards first appeared
\cite{SkateboardingEncyclobeadia}. It has since developed into a worldwide
phenomenon, with its popularity skyrocketing, after gaining recognition as an
official sport in the 2020 Tokyo Olympics \cite{SkatebaordingOlimpics}.
Skateboarding comprises the dynamic activities of riding a skateboard and
skillfully performing a repertoire of tricks, manifesting as a popular and
exhilarating “extreme sport”. 
\paragraph{}
This dynamic sport encompasses various disciplines and styles of riding, each of
them offering unique challenges for skateboarders to explore. Two of the most
prominent styles are “vert” and “Street”, vert skateboarding revolves around
riding on specialised obstacles, namely, half-pipes and ramps, emphasising areal
manoeuvres. Whilst, street skateboarding transpires in urban environments,
utilising a diverse array of obstacles that can be found outdoors, including
stairs, rails, ledges, gaps or flat ground for skaters to showcase their
creativity. \cite{skateStyles}.

\paragraph{}
Skateboard tricks are the heart and soul of skateboarding. These tricks
originate from the dynamic orchestration of rotations and revolutions of a
skateboard along various axes emphasising the significance of precise placement
of a skateboarder's feet to initiate these rotations. These tricks serve as
excellent examples of how the skateboarder's body and skateboard work in perfect
harmony. Some common skateboard tricks include:

\begin{itemize}
    \item \textbf{Ollie:} One of the first tricks beginners learn. Where the
    skateboarder pops the tail of the board while sliding their foot across the
    board, causing the board to level out in the air, used to jump over
    obstacles.
    \item \textbf{Kickflip:} A trick where the skateboarder flips the board
    under their feet while jumping, making it spin 360\textdegree around the
    x-axis.
    \item \textbf{360 kickflip:} A combination of a kickflip and a
    360\textdegree board rotation around the y-axis.
\end{itemize}
Skateboarders continually innovate and come up with new trick combinations,
contributing to the dynamic nature of the sport.
\paragraph{}

\section{Problem Specification}

In skateboarding competitions, the current method of identifying tricks relies
heavily on judges verabally stating stating them during live broadcasts. This
approach lacks a digital overlay that could display the performed trick for the
viewers, leading to a reliance on subjective judgement. Such subjectivity can
lead to scoring disparities and conflicts, undermining competition fairness.
Furthermore,this absence of an objective method, obstructs skaters' ability to
receive real-time feedback, which could be beneficial for skill development. The
advent of an AI-based system that delivers a dependable and efficient method of
trick recognition will both revolutionise this aspect and contribute to the lack
of study in this area. 


\section*{General Aims and Objectives}
The aims and objectives are to design and develop a dependable
Artificial Intelligence (AI) model that can identify various skateboard tricks
directly from videos. An initial focus will be placed on the recognition of three
different tricks namely, kickflips, ollies and shuvits. To reach this goal the
appropriate ML techniques must be utilised to overcome basic computer vision
challenges, such as varying camera angles, lighting conditions and the
fast-paced nature of skateboard tricks. By addressing these challenges, this
project aims to establish a usable tool that skateboarders, coaches, and
skateboard competitions can use.
\subsection{List of deliverables}
\begin{itemize}
    \item The 
    \item A trained model with favourable accuracy
    \item The results of the experiments and their visualisations
\end{itemize}

\section{Brief overview of the literature} 

Recent advancements in the ability to recognise human actions in videos have
significantly impacted various fields, ranging from the medical sector
\cite{3DhumanActionDetectionForHealthCareSystems}, to security monitoring
\cite{HumanActivityRecognitionSecurityAndMonitoring}. Particularly noteworthy is
its application in the development of an AI-based skateboard trick classifier,
an area that has seen limited research.
\paragraph{}

In this emergent field,leveraging the techniques of activity recognition from
video has led to two primary techniques among researchers. The first technique
involves utilising signals obtained from skateboard-mounted accelerometers or
signals that have been artificially generated based on findings of prior
studies. These signals are then fed into a study-dependent model for
classification, as outlined in
\cite{skateboardClassificationTransferLearningPipelinesAccelermetry} and
\cite{skateboardTrickClassifierUsingAccelerometryAndML}. The second approach
employs computer vision techniques,leveraging video footage of skateboard tricks
to train and refine models for accurate trick identification, as depicted by the
studies \cite{skatePaper1} and \cite{SkateboardAIPaper}.

\subsection{Accelerometry approach}
\subsection{Computer Vision approach}

The paper \cite{skatePaper1} leverages a custom data set, comprising videos
capturing the execution of five distinct skateboard tricks, each attempted five
times. Each video spans a duration of two to three seconds, yielding a total of
750 images by extracting 30 frames per video. This study made use of data
augmentation techniques to further expand their dataset. Consequently, they
introduced an additional 2,250 images, achieving a total of 3,000 images in
their data set. On the other hand, Chen \cite{SkateboardAIPaper} compiled a
comprehensive data set by collecting videos from multiple platforms including
YouTube, Twitter and Instagram. Furthermore, the researcher trained the model
using 15 fundamental tricks commonly observed in competitive settings. They
collected 50 videos per trick, summing up to a total number of 750 videos. Of
these, 45 videos per trick were allocated for training, and the remaining 5 were
reserved for validation.

Data augmentation techniques are popular amongst studies with relatively limited
datasets. Techniques such as such as flipping, rotating, scaling and colour
manipulation not only enhance the size of the original dataset but also lower
the likelihood of the model overfitting
\cite{DataAugmentationCanImproveRobustness}. The study \cite{skatePaper1}
utilises three rotation augmentation techniques: horizontal rotation, positive
90\textdegree rotation and negative 90\textdegree rotation.
\paragraph{}
Recurrent Neural Networks (RNNs) and Long Short-Term memory networks (LSTMs) are
popular architecures due to their capabilities in modelling the dynamic
relationships in sequential data \cite{UnderstandingLSTM}. In the student
abstract by Hanciao Chen \cite{SkateboardAIPaper}, extensive experimentation is
conducted using diverse models, exploring various combinations of CNN-LSTM and
CNN-BiLSTM including the integration of attention and transfer learning
approaches. This study further documents and analyzes important metrics such as
training time, training accuracy and validation accuracy for each model
experimented on. Among these, the top three models that stood out in terms of
validation accuracy were the ResNet50 with Attention and BiLSTM (84\%), ResNet50
with BiLSTM (81\%) and ResNet50 with LSTM (80\%).Chen's study provides valuable
insight on the applicaiton of diverse models in the application of activity
recognition in skateboarding.

%Talk about the other paper




\section{Planned methodologies}

In terms of data collection, this project will utilise the open source datasets
provided by \cite{lightningdrop2020skateboardml} and \cite{SkateboardAIPaper}.
Moreover, to ensure an unbiased and rigorous evaluation of the model's
performance, this study will incorporate a dataset, consisting of skateboard
tricks performed and recorded specifically for this study.

Initially w
by 




 

\section{Project timeline}
