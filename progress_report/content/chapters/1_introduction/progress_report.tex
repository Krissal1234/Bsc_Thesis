% \chapter{Progress Report}%
% \label{chp:Progress Report}
% include in intro:

% - show the importance of the topic by discussing the underutilisation of AI in sports
% - references to current research - mention some papers that have created skateboard ai's

% - explain the limits of the ai - developing an ai that can recognise atelast two different tricks effectively

% - how the paper will be organised
%  - e.g. An assessment will first be made of the performance of the privatized industries themselves, on an individual basis, and then the performance of the economy as a whole will be examined.
 
%  - thesis statement - this is that sentence or two that contain the focus of your essay and tell your reader what the essay’s aim is.
 
%  - CARS model
%  	MOVE 1: Establishing a Territory
% 	MOVE 2: Establishing a Niche 
% 	MOVE 3: Occupying the Niche
 
 
%  PLAN
 
%  - Definitions
 
%  - references to past research
 
 
%  - emphasizing hte importance of the topic
 
%  - I will establish the niche using the underutilisation of ai in sports and how under researched ai in skateboarding is
 
 
%  - show importnace of the topic
%  - explain limits of the research
%  - how the paper will be organised
 
%  - Thesis statement
 
\section*{Introduction}
Skateboarding dates back to the 1940s when handmade skateboards first appeared
\cite{SkateboardingEncyclobeadia}. It has since developed into a worldwide
phenomenon, with its popularity skyrocketing, after gaining recognition as an
official sport in the 2020 Tokyo Olympics \cite{SkatebaordingOlimpics}.
Skateboarding comprises the dynamic activities of riding a skateboard and
skillfully performing a repertoire of tricks, manifesting as a popular and
exhilarating “extreme sport”. 
\paragraph{}
This dynamic and multifaceted sport encompasses various disciplines and styles
of riding, each of them offering unique challenges for skateboarders to delve
into. Two of the most prominent styles are “vert” and “Street”, vert
skateboarding revolves around riding on specialised structures known as
half-pipes or ramps, emphasising areal manoeuvres. Whilst, street skateboarding
transpires in urban environments, utilising a diverse array of obstacles,
including stairs, rails, ledges, gaps or flat ground for skaters to showcase
their creativity and versatility through various techniques \cite{skateStyles}.

\subsection{Skateboard Tricks}
\paragraph{}
Skateboard tricks are the heart and soul of skateboarding. These tricks
originate from the dynamic orchestration of rotations and revolutions of a
skateboard along various axes emphasising the significance of precise placement
of a skateboarder's feet to initiate these rotations. These tricks serve as
excellent examples of how the skateboarder's body and skateboard work in perfect
harmony. Some common skateboard tricks include:

\begin{itemize}
    \item \textbf{Ollie:} One of the first tricks beginners learn. Where the
    skateboarder pops the tail of the board while sliding their foot across the
    board, causing the board to level out in the air, used to jump over
    obstacles.
    \item \textbf{Kickflip:} A trick where the skateboarder flips the board
    under their feet while jumping, making it spin 360\textdegree around the
    x-axis.
    \item \textbf{360 kickflip:} A combination of a kickflip and a
    360\textdegree board rotation around the y-axis.
\end{itemize}

Skateboarders continually innovate and come up with new trick combinations,
contributing to the dynamic nature of the sport.


\section*{General Aims and Goals}
The primary goal of this project is to design and develop an efficient
Artificial Intelligence (AI) model that can recognise and distinguish differnent
skateboard tricks from video footage. Initially, the focus will be placed on the
recogntion of tricks such as kickflips, ollies and shuvits. This approach
entails applying the appropriate Machine Learning (ML) techniques to overcome
commonly encountered challenges in computer vision. By addressing these
challenges, such as the fluctuation of camera angles, lighting conditions and
the complex dynamics of skateboard manoevers, this project seeks to develop an
accessible tool that may be utilised by skateboarders, coaches or skateboard
contests.