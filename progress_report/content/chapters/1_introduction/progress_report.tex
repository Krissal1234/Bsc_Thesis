% \chapter{Progress Report}%
% \label{chp:Progress Report}
% include in intro:

% - show the importance of the topic by discussing the underutilisation of AI in sports
% - references to current research - mention some papers that have created skateboard ai's

% - explain the limits of the ai - developing an ai that can recognise atelast two different tricks effectively

% - how the paper will be organised
%  - e.g. An assessment will first be made of the performance of the privatized industries themselves, on an individual basis, and then the performance of the economy as a whole will be examined.
 
%  - thesis statement - this is that sentence or two that contain the focus of your essay and tell your reader what the essay’s aim is.
 
%  - CARS model
%  	MOVE 1: Establishing a Territory
% 	MOVE 2: Establishing a Niche 
% 	MOVE 3: Occupying the Niche
 
 
%  PLAN
 
%  - Definitions
 
%  - references to past research
 
 
%  - emphasizing hte importance of the topic
 
%  - I will establish the niche using the underutilisation of ai in sports and how under researched ai in skateboarding is
 
 
%  - show importnace of the topic
%  - explain limits of the research
%  - how the paper will be organised
 
%  - Thesis statement
 
\section*{Introduction}
Skateboarding dates back to the 1940s when handmade skateboards first appeared
\cite{SkateboardingEncyclobeadia}. It has since developed into a worldwide
phenomenon, with its popularity skyrocketing, after gaining recognition as an
official sport in the 2020 Tokyo Olympics \cite{SkatebaordingOlimpics}.
Skateboarding comprises the dynamic activities of riding a skateboard and
skillfully performing a repertoire of tricks, manifesting as a popular and
exhilarating “extreme sport”. 
\paragraph{}
This dynamic sport encompasses various disciplines and styles of riding, each of
them offering unique challenges for skateboarders to explore. Two of the most
prominent styles are “vert” and “Street”, vert skateboarding revolves around
riding on specialised obstacles, namely, half-pipes and ramps, emphasising areal
manoeuvres. Whilst, street skateboarding transpires in urban environments,
utilising a diverse array of obstacles that can be found outdoors, including
stairs, rails, ledges, gaps or flat ground for skaters to showcase their
creativity. \cite{skateStyles}.

\paragraph{}
Skateboard tricks are the heart and soul of skateboarding. These tricks
originate from the dynamic orchestration of rotations and revolutions of a
skateboard along various axes emphasising the significance of precise placement
of a skateboarder's feet to initiate these rotations. These tricks serve as
excellent examples of how the skateboarder's body and skateboard work in perfect
harmony. Some common skateboard tricks include:

\begin{itemize}
    \item \textbf{Ollie:} One of the first tricks beginners learn. Where the
    skateboarder pops the tail of the board while sliding their foot across the
    board, causing the board to level out in the air, used to jump over
    obstacles.
    \item \textbf{Kickflip:} A trick where the skateboarder flips the board
    under their feet while jumping, making it spin 360\textdegree around the
    x-axis.
    \item \textbf{360 kickflip:} A combination of a kickflip and a
    360\textdegree board rotation around the y-axis.
\end{itemize}
Skateboarders continually innovate and come up with new trick combinations,
contributing to the dynamic nature of the sport.
\paragraph{}


\section*{General Aims and Goals}
The primary goal of this project is to design and develop a dependable
Artificial Intelligence (AI) model that can identify various skateboard tricks
directly from video. An initial focus will be placed on the recognition three
different tricks namely, kickflips, ollies and shuvits. To reach this goal the
appropriate ML techniques must be utilised to overcome basic computer vision
challenges, such as varying camera angles, lighting conditions and the
fast-paced nature of skateboard tricks. By addressing these challenges, this
project aims to establish a usable tool that skateboarders, coaches, and
skateboard competitions can use.

\section{Brief and represented literature review} 
There has been very little research in recent years on the development of a
skateboard trick classifier using Artificial Intelligence (AI). In this emergent
field, two primary techniques have fostered prominence among researchers. The
first technique involves utilising signals obtained from skateboard-mounted
accelerometers or signals that have been artificially generated based on
findings of prior studies to reproduce tricks. These signals are then fed into a
study-dependent model for classification, as outlined in
\cite{skateboardClassificationTransferLearningPipelinesAccelermetry} and
\cite{skateboardTrickClassifierUsingAccelerometryAndML}. The second approach
employs computer vision techniques,leveraging video footage of skateboard tricks
to train and refine models for accurate trick identification, as depicted by the
studies \cite{skatePaper1} and \cite{SkateboardAIPaper}.

\subsection{Accelerometry approach}
\subsection{Computer Vision appraoch}

\section{Planned methodologies}
 
\section{Problem Specification}

The current lack of a digital and objective mechanism for identifying skateboard
tricks during competitions and practice sessions is a significant issue. Judges
in live-streamed skateboarding competitions currently announce tricks verbally, without the
support of a digital overlay to display the performed trick for the viewers.
This reliance on subjective judgement can lead to scoring disparities and
conflicts, weakening competition fairness. Furthermore,this absence of an
objective method, obstructs skaters' ability to receive real-time feedback, which
could be beneficial for skill development. The advent of an AI-based system that
delivers a dependable and efficient method of trick recognition will both
revolutionise this aspect and contribute to the lack of study in this area. 

\section{Project timeline}
