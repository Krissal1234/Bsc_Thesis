% \chapter{Progress Report}%
% \label{chp:Progress Report}
% include in intro:

% - show the importance of the topic by discussing the underutilisation of AI in sports
% - references to current research - mention some papers that have created skateboard ai's

% - explain the limits of the ai - developing an ai that can recognise atelast two different tricks effectively

% - how the paper will be organised
%  - e.g. An assessment will first be made of the performance of the privatized industries themselves, on an individual basis, and then the performance of the economy as a whole will be examined.
 
%  - thesis statement - this is that sentence or two that contain the focus of your essay and tell your reader what the essay’s aim is.
 
%  - CARS model
%  	MOVE 1: Establishing a Territory
% 	MOVE 2: Establishing a Niche 
% 	MOVE 3: Occupying the Niche
 
 
%  PLAN
 
%  - Definitions
 
%  - references to past research
 
 
%  - emphasizing hte importance of the topic
 
%  - I will establish the niche using the underutilisation of ai in sports and how under researched ai in skateboarding is
 
 
%  - show importnace of the topic
%  - explain limits of the research
%  - how the paper will be organised
 
%  - Thesis statement
 
\section*{Introduction}
\subsection*{Introduction to Skateboarding and its Evolution}
\paragraph{}
Skateboarding dates back to the 1940s when handmade skateboards first appeared
\cite{SkateboardingEncyclobeadia}. It has since developed into a worldwide
phenomenon, with its popularity skyrocketing, after gaining recognition as an
official sport in the 2020 Tokyo Olympics \cite{SkatebaordingOlimpics}.
Skateboarding comprises the dynamic activities of riding a skateboard and
skillfully performing a repertoire of tricks, manifesting as a popular and
exhilarating “extreme sport”. 
\paragraph{}
This dynamic sport encompasses various disciplines and styles of riding, each of
them offering unique challenges for skateboarders to explore. Two of the most
prominent styles are “vert” and “Street”, vert skateboarding revolves around
riding on specialised obstacles, namely, half-pipes and ramps, emphasising areal
manoeuvres. Whilst, street skateboarding transpires in urban environments,
utilising a diverse array of obstacles that can be found outdoors, including
stairs, rails, ledges, gaps or flat ground for skaters to showcase their
creativity. \cite{skateStyles}.


\subsection*{Introduction to the Area}
\paragraph{}
The current method of identifying tricks in skateboard competitions relies
heavily on judges verabally stating them during live broadcasts. This approach
lacks a digital overlay that could display the performed trick for the viewers,
leading to a reliance on subjective judgement. Such subjectivity can lead to
scoring disparities and conflicts, undermining competition fairness.
Furthermore, this absence of an objective method of identification, obstructs
skaters' ability to receive real-time feedback, which could be beneficial for
skill development. The advent of an AI-based system that delivers a dependable
and efficient method of trick recognition will both revolutionise this aspect
and contribute to the lack of study in this area. 

\section*{Scope and Motivation}



\section*{Brief overview of the literature} 

Recent advancements in the ability to recognise human actions in videos have
significantly impacted various fields, ranging from the medical sector to
surveillance systems, \cite{3DhumanActionDetectionForHealthCareSystems},
\cite{HumanActivityRecognitionSecurityAndMonitoring}. Particularly noteworthy is
its application in the development of an AI-based skateboard trick classifier,
an area that has seen limited research.
\paragraph{}

In this emergent field, leveraging the techniques of activity recognition from
video has led to two primary methodologies among researchers. The first technique
involves utilising signals obtained from skateboard-mounted accelerometers or
signals that have been artificially generated based on findings of prior
studies. These signals are then fed into a study-dependent model for
classification, as outlined in
\cite{skateboardClassificationTransferLearningPipelinesAccelermetry} and
\cite{skateboardTrickClassifierUsingAccelerometryAndML}. The second approach
employs computer vision techniques, leveraging video footage of skateboard tricks
to train and refine models for accurate trick identification, as depicted by the
studies \cite{skatePaper1} and \cite{SkateboardAIPaper}.

\subsection*{Accelerometry approach}
The study \cite{skateboardClassificationTransferLearningPipelinesAccelermetry},
makes use of a custom dataset comprising of six skateboard tricks most commonly
executed in competitive events. Amateur skateboarders performed each trick five
times on a modified skateboard equipped with an Intertial Measurement Unit (IMU)
to record the signals produced. The researchers capture six
signals for each trick, including linear accelerations along the x, y, and z
axes (aX, aY, aZ) and angular accelerations along the same axes (gX, gY, gZ).
They then opt for the unique appraoch of assembling all six signals into a
single image corresponding to one trick, employing two input image transformations,
raw transformations (RAW) and Continuous Wavelet Transform (CWT).
% go on and talk about results ---CHECK becuase results are likely not accurate

\paragraph{}
On the other hand, in the study
\cite{skateboardTrickClassifierUsingAccelerometryAndML} researchers utilised
MATLAB 2015 and Signal Processing Toolbox to artificially generate 543 signals,
based on prior research in this area. These signals were then categorised into five
distinct classes representing different skateboard tricks, each with a varied
number of samples ranging from 30 to 50 per class, across three axes (X,Y and
Z). 


\subsection*{Computer Vision approach}

The paper \cite{skatePaper1} leverages a custom data set, comprising videos
capturing the execution of five distinct skateboard tricks, each attempted five
times. Each video spans a duration of two to three seconds, yielding a total of
750 images by extracting 30 frames per video. This study made use of data
augmentation techniques to further expand their dataset. Consequently, they
introduced an additional 2,250 images, achieving a total of 3,000 images in
their data set. On the other hand, Chen \cite{SkateboardAIPaper} compiled a
comprehensive data set by collecting videos from multiple platforms including
YouTube, Twitter and Instagram. Furthermore, the researcher trained the model
using 15 fundamental tricks commonly observed in competitive settings. They
collected 50 videos per trick, summing up to a total number of 750 videos. Of
these, 45 videos per trick were allocated for training, and the remaining 5 were
reserved for validation.

\paragraph{}
Data augmentation techniques are popular amongst studies with relatively limited
datasets. Techniques such as such as flipping, rotating, scaling and colour
manipulation not only enhance the size of the original dataset but also lower
the likelihood of the model overfitting
\cite{DataAugmentationCanImproveRobustness}. The study \cite{skatePaper1}
utilises three rotation augmentation techniques: horizontal rotation, positive
90\textdegree rotation and negative 90\textdegree rotation.

\paragraph{}
Recurrent Neural Networks (RNNs) and Long Short-Term memory networks (LSTMs) are
popular architecures due to their capabilities in modelling the dynamic
relationships in sequential data \cite{UnderstandingLSTM}. In the student
abstract by Hanciao Chen \cite{SkateboardAIPaper}, extensive experimentation is
conducted using diverse models, exploring various combinations of CNN-LSTM and
CNN-BiLSTM including the integration of attention and transfer learning
approaches. This study further documents and analyzes important metrics such as
training time, training accuracy and validation accuracy for each model
experimented on. Among these, the top three models that stood out in terms of
validation accuracy were the ResNet50 with Attention and BiLSTM (84\%), ResNet50
with BiLSTM (81\%) and ResNet50 with LSTM (80\%). Chen's study provides valuable
insight on the applicaiton of diverse models in the application of activity
recognition in skateboarding.


\section*{Aims and Objectives}
\paragraph{}
The aims and objectives of this project are to design and develop a dependable
Artificial Intelligence (AI) model that can identify various skateboard tricks
directly from videos. An initial focus will be placed on the recognition of three
distinct tricks namely, kickflips, ollies and shuvits. To reach this goal the
appropriate ML techniques must be utilised to overcome basic computer vision
challenges, such as varying camera angles, lighting conditions and the
fast-paced nature of skateboard tricks. By addressing these challenges, this
project aims to establish a usable tool that can be used by skateboarders, coaches, and
skateboard competitions.


\section*{List of deliverables}
\begin{itemize}
    \item The prototype of the AI model
    \item A trained model with a accuracy metrics
    \item Documentation of experiments and their visualisations
\end{itemize}


\section*{Methodology and Evaluation Strategy}
\paragraph{}
In terms of data collection, this project will utilise a combination of the open
source datasets provided by \cite{lightningdrop2020skateboardml} and
\cite{SkateboardAIPaper}. Moreover, to ensure an unbiased and rigorous
evaluation of the model's performance, this study will incorporate a dataset,
consisting of skateboard tricks performed and recorded specifically for this
study. 
\paragraph{}
In the course of my thesis research, I initially began by experimenting with
various preprocessing techniques such as applying frame extraction to videos and
employing object detection to selectively crop each frame, focusing on
encapsulating only the skater. This approach was designed to remove any
unneccessary overhead in each frame that might otherwise impede the learning
process. 
\paragraph{}
I commenced the development phase using Python, opting for YOLO (You Only Look
Once) for the object detection framework. The YOLO training process involved
gathering a diverse dataset and meticulously labelling each image featuring a
skateboarder by drawing an accurate bounding box around each subject. The trined
object detection model takes an image as an input and returns the coordinates of
the bounding box encasing the estimated position of the skater in frame, used to
crop the image accordingly. 

\paragraph{}
After completing the experimentation phase with various preprocessing techniques
and creating a temporary preprocessing mechanism, I shifted my focus towards the
development of a baseline model. Establishing a baseline model creates a
benchmark for all future modifications and enhancements to it. This foundational
model, characterised by its ConvLSTM architecture, is specifically designed for
sequential data processing. It begins with a ConvLSTM2D input layer, which
features a 3x3 kernel size and employs a 'tanh' activation function.
Subsequently this model incorporates a MaxPooling3D layer with a pool size of
1x2x2, complemented by TimeDistributed Dropout layers to prevent overfitting.
The architecture consistently follows this pattern, progressively increasing the
number of filters, finally ending with a Dense layer. After training, the
baseline model achieved a favourable but improvable validation accuracy of 72%. 

\subsection*{Work Plan}
I plan to further refine the primary model and aim to achieve a more efficient
and accurate model. This will involve applying various preprocessing and
training techniques, including but not limited to data augmentation and the
'leave-one-out' approach. Additionally I intend to experiment with transfer
learning methods, incorporating models like ResNet or VGG for feature extraction
and explore adjustments in architecture and hyperparameter modifications.


\section*{Project timeline}
\begin{figure}[h]
    \centering
    \includegraphics[width=1\textwidth]{content/chapters/1_introduction/figures/newGantt.png}
    \caption{Estimated project timeline.}
    \label{fig:my_label}
\end{figure}