\clearpage
\chapter{Literature Review}


\section{Activity Recognition}
Building on the foundational understanding of activity recognition defined in
the background, it's important to acknowledge the impact it has on
various sectors. The recent advancements in recognising human actions in videos
have have not only revolutionalised sectors such as healthcare and security
\cite{3DhumanActionDetectionForHealthCareSystems},
\cite{HumanActivityRecognitionSecurityAndMonitoring}, but also hold extensive
potential in the realm of sports. K. Host and M. Ivašić-Kos in
\cite{HARinSportsComputerVision} along with Wu et al. in
\cite{Asurveyonvideoactionrecognitioninsports:Datasetsmethodsandapplications}
further elaborate on this potential, highlighting the numerous applications and
importance of activiy recognition.
\paragraph{} 
In the study by Beddiar et al. \cite{VisionBasedHARASurvey}, two main streams of
human computer interaction (HCI) technologies are emphasized: Contact-based and
Vision-based systems. The authors categorise contact-based HCI as those
technologies that require physical user interaction through mediums such as
accelerometers, wearable sensors and multi-touch interfaces. 
% Complimenting this perspective, the study by Cook et al. \cite{SensorBasedActivityRecognition}
% delves into the the specifics of sensor technology, exploring how various sensor
% types are significant in the field of activity recognition.
Alternatively, the authors describe vision-based methods as the simplification
of HCI due to more natural human communication, eliminating the need for
physical contact or equipment. These methods use image and video data to
recognise human activities offering an advantage in terms of societal acceptance
and useability.
\paragraph{}
While both contact and vision-based systems have their merits, this study will
specifically focus on vision-based techniques and their application in the
development of a skateboard trick classifier. These methods, which utilise image
and video data to recognise human activities are selected due to their
societal acceptability and their applicability in sports broadcasts.
\subsection{Challenges in this field}
\paragraph{}
The domain of Activity recognition comes with many challenges as depicted by
Zhang et al. (2017)
\cite{ReviewOfHumanActivityRecognitionUsingVisionBasedMethod}. The researchers
argue that while in some certain scenarios such as surveillance systems utilise
static cameras, most scenarios that would benefit from Activity recognition
adopt dynamic recording devices such as sports event broadcasts or mobile
phones. These dynamic devices introduce a significant level of complexity due to
the complex dynamic backgrounds typically present in recorded video footage.
Zhang et al. also points out specific difficulties posed by long-distance and
low-quality videos, often encountered in environments like crowded public spaces
and sports events. The camera distance, results in smaller subjecs, making
detailed analysis of human movements more challenging while lower
quality videos further complicate the task for Human Activity Recognition (HAR)
systems. 
\paragraph{}
The challenges highlighted by Zhang et al.
\cite{ReviewOfHumanActivityRecognitionUsingVisionBasedMethod} in the domain of
activity recognition are directly relavant to the development of a skateboard
trick classifier. In scenarios like televised skateboarding events, skaters may
appear relatively small to accommodate the entire skatepark, This factor along
with the complexity of the background as a result of dynamic recording poses a
challenge for trick recognition in live broadcasts. Furthermore, if a skateboard
trick classifier is intended for home use, then it may encounter videos of lower
quality further complicating the task of trick recognition. Addressing these
challenges is crucial for the development of a skateboard trick classifier that
performs well in real-world conditions.

\subsection{Activity Recognition Techniques}


%Maybe first mention that ML classification can be used, but more recently, deep
%learning has been favourable HARinSports K. host, then move onto different
%studies and how they tackled activity recognition by discussing the evolution
%from machine learning to deep learning in activity recognition, and why deep
%learning might be more effective for your study

% align with the insights from the
% paper by K. Host and M. Ivašić-Kos \cite{HARinSportsComputerVision}. This paper
%While Contact based is good, this study will be focusing on computer vsision

\section{Preprocessing techniques}
Preprocessing is a crucial step
As a result of the limited research in the advent of a skateboard trick
classifier, there is a significan lack of open-source datasets featuring
skateboard tricks. This lack of data presents an opportunity to employ data
augmentation techniques, especially valuable in studies with limited data.
Teqniques such as flipping, rotating, scaling and colour manipulation not only
artificially enhance the size of the original dataset but also lower the
likelihood of the model overfitting \cite{DataAugmentationCanImproveRobustness},
\cite{AnOverviewOfOverfittingAndItsSolutions}. In the realm of Skateboard trick
classifiers, Shapiee et al (2020) \cite{skatePaper1} effectively employed data
augmentation techniques to expand their dataset, demonstrating the application
of these methods in improving model performance for trick classification.
\paragraph{}
After establishing the role of data augmentation in addressing the lack of data
available
%look into optical flow due to the dynamic notion of skateboard tricks - paper HARInSports
%look into Histogram of Oriented Gradient (HOG) 3D features,
\section{Advancements in Skateboard Trick Classification}

\paragraph{}
In the emergent field of skateboard trick classification, leveraging activity
recognition techniques from a video has led to two primary methodologies among
researchers. The first technique involves utilising signals obtained from
skateboard-mounted accelerometers or signals artificially generated based on the
findings of prior studies. These signals are then fed into a study-dependent
model for classification, as outlined in
\cite{skateboardClassificationTransferLearningPipelinesAccelermetry} and
\cite{skateboardTrickClassifierUsingAccelerometryAndML}. The second approach
employs computer vision techniques, leveraging video footage of skateboard
tricks to train and refine models for accurate trick identification, as depicted
by the studies \cite{skatePaper1} and \cite{SkateboardAIPaper}. 

\subsection{Accelerometry approach}
\paragraph{}
The study by Abdullah et al (2021)
\cite{skateboardClassificationTransferLearningPipelinesAccelermetry}, makes use
of a custom dataset comprising of six skateboard tricks most commonly executed
in competitive events. Amateur skateboarders performed each trick five times on
a modified skateboard equipped with an Inertial Measurement Unit (IMU) to record
the signals produced. The researchers capture six signals for each trick,
including linear accelerations along the x, y, and z axes (aX, aY, aZ) and
angular accelerations along the same axes (gX, gY, gZ). They then opt for the
unique approach of concatenating all six signals onto a single image
corresponding to one trick, employing two input image transformations: raw data
(RAW) and Continuous Wavelet Transform (CWT). 
\paragraph{}
With the application of six transfer learning models on this data, Abdullah et
al. \cite{skateboardClassificationTransferLearningPipelinesAccelermetry} reports
exceptionally high accuracies, achieving a 100\% test accuracy over multiple
models. While these results are remarkable, very high levels of accuracy are
rare in ML applications and are typically associated with models that may be
overfitting the data. Recognising the rarity of such high accuracies, this study
will take these findings into consideration and efforts will be made to ensure a
robust model by employing techniques to avoid overfitting such as early-stopping
and the use of a diverse dataset \cite{AnOverviewOfOverfittingAndItsSolutions}.


% the results show that RAW and CWT input
% images on MobileNet, MobileNetV2 and ResNet101 models achieved favourable
% accuracy. However, the CWT-MobileNet-Optimised SVM pipeline was deemed the best
% due to its reduction in computational time.

\paragraph{}
The study by Corrêa et al (2017)
\cite{skateboardTrickClassifierUsingAccelerometryAndML}, obtained their sample
data by artificially generating 543 signals based on prior research, utilising
tools such as MATLAB 2015 and Signal Processing Toolbox. These signals were then
categorised into five distinct classes representing different skateboard tricks,
each with various samples ranging from 30 to 50 per class, across three axes (X,
Y and Z). This study developed and validated individual Artificial Neural
Networks (ANNs) for each axis, as well as the combination of the three: ANN XYZ,
displaying the potential of Neural Networks to categorise multidimensional
skateboard tricks. The ANNs are all multilayer feed-forward neural networks
(MFFNNs), structured into three distinct layers. They feature an input layer
with 82 neurons, a hidden layer, comprised of 23 neurons utilising a tan-sigmoid
transfer function and an output layer consisting of 5 neurons with a softmax
function. Finally, the study achieved high accuracies, with ANNs X, Y and Z
achieving accuracies of 94.8\%, 96.7\% and 98.7\%, respectively, while the
combined ANN XYZ achieved an accuracy of 92.8\%. 


\subsection{Computer Vision Approach}
\paragraph{}
The paper by Shapiee et al (2020) \cite{skatePaper1} leverages a custom data set
comprising videos capturing the execution of five distinct skateboard tricks,
each attempted five times. Each video spans two to three seconds, yielding a
total of 750 images by extracting 30 frames per video. This study made use of
data augmentation techniques to expand their dataset further. Consequently, they
introduced an additional 2,250 images, achieving 3,000 images in their data set.
On the other hand, Chen (2023) \cite{SkateboardAIPaper} compiled a comprehensive
data set by collecting videos from multiple platforms, including YouTube,
Twitter and Instagram. Furthermore, Chen trained the model using 15 fundamental
tricks commonly observed in competitive settings. The researcher collected 50
videos per trick, summing up to a total number of 750 videos. Of these, 45
videos per trick were allocated for training, and the remaining 5 were reserved
for validation. 

\paragraph{}
%Maybe move to prerpcessing
% Data augmentation techniques are popular in studies with relatively limited
% datasets. Techniques such as flipping, rotating, scaling and colour manipulation
% not only enhance the size of the original dataset but also lower the likelihood
% of the model overfitting \cite{DataAugmentationCanImproveRobustness},
% \cite{AnOverviewOfOverfittingAndItsSolutions}. 
%Mention more here !!!!
The paper by Shapiee et al. \cite{skatePaper1} utilises data augmentation
techniques and applies three rotation augmentation techniques: horizontal rotation,
positive 90°rotation and negative 90°rotation. The researchers experimented on
three Transfer learning models: MobileNet, NASNetMobile and NASNetLarge, each
evaluated using a k-Nearest Neighbor (k-NN) classifier. As a result, the models
demonstrated impressive classification accuracies, with MobileNet achieving
95\%, NASNetMobile 92\% and NASNetLarge 90\%. 

\paragraph{}
Recurrent Neural Networks (RNNs) and Long Short-Term Memory Networks (LSTMs) are
popular architectures due to their capabilities in modelling the dynamic
relationships in sequential data \cite{UnderstandingLSTM}. In the student
abstract by Hanciao Chen \cite{SkateboardAIPaper}, extensive experimentation is
conducted using diverse models, exploring various combinations of CNN-LSTM and
CNN-BiLSTM architectures. The study also incorporated attention mechanisms and
explored transfer-based methods for action recognition. 
%Add more information on 
This study further
documents and analyses important metrics such as training time, training
accuracy and validation accuracy for each model experimented on. Among these,
the top three models that stood out in terms of validation accuracy were the
ResNet50 with Attention and BiLSTM (84\%), ResNet50 with BiLSTM (81\%) and
ResNet50 with LSTM (80\%). Chen's study provides valuable insight into the
application of diverse models in activity recognition in skateboarding. 

\section{Model Evaluation and Challenges}

% Discuss the evaluation metrics used in these studies, like accuracy, precision, recall, etc.
% Address challenges like overfitting, especially in the context of studies reporting unusually high accuracies.

\section{Conclusion of the Literature Review}