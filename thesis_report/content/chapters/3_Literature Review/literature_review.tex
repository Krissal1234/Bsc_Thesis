\clearpage
\section{Literature Review}

\subsection{Data Collection}
\paragraph{}
The assortment of data stands out as the foundation of this study, it lends
significance to the results as the quality and quantity of data have a direct
relation with the capacity of AI algorithms to learn and generalise effectively. 

\paragraph{}
The study \cite{skatePaper1} leverages a custom data set, comprising videos
capturing the execution of five distinct skateboard tricks, each attempted five
times. Each video spans a duration of two to three seconds, yielding a total of
750 images by extracting 30 frames per video. Notably, each skateboard manoeuvre
was recorded against a uniform white backdrop, maintaining a consistent distance
of 1.26 meters between the skateboarder and the camera. In this study, the
researchers applied data augmentation techniques to expand their initial data
set in order to reduce the potential for model over-fitting noted by
\cite{DietterichOverfittingAI}. This paper utilises three rotation augmentation
techniques: horizontal rotation, positive 90\textdegree rotation and negative
90\textdegree rotation. Consequently, they introduced an additional 2,250
images, achieving a total of 3,000 images in their data set.

\paragraph{}
On the other hand, Chen \cite{SkateboardAIPaper} compiled a comprehensive data
set by collecting videos from multiple platforms including YouTube, Twitter and
Instagram. Furthermore, the researcher trained the model using 15 fundamental
tricks commonly observed in competitive settings. They collected 50 videos per
trick, summing up to a total number of 750 videos. Of these, 45 videos per trick
were allocated for training, and 5 were reserved for validation.

\subsection{Data Preprocessing}
\paragraph{}
An AI model cannot rely solely on raw data to achieve a favourable accuracy.
Instead, it benefits from the utilisation of preprocessing techniques that
enhance features and guarantee that data is formatted consistently
\cite{DataPreprocessingForImageClassificationByCNN}.







