\chapter{Introduction}%
\label{chp:introduction}

In progress --- incorporate these paragraphs

Skateboarding dates back to the 1940s when handmade skateboards first appeared
\cite{SkateboardingEncyclobeadia}. It has since developed into a worldwide
phenomenon, with its popularity skyrocketing, after gaining recognition as an
official sport in the 2020 Tokyo Olympic Games \cite{SkatebaordingOlimpics}.
Skateboarding comprises the dynamic activities of riding a skateboard and
skilfully performing a repertoire of tricks, manifesting as a popular and
exhilarating “extreme sport”. 

This dynamic sport encompasses various disciplines and riding styles, each
offering unique challenges for skateboarders to explore. Two of the most
prominent styles are “vert” and “street.” Vert skateboarding revolves around
riding on specialised obstacles, namely, half-pipes and ramps, emphasising
aerial manoeuvres. Street skateboarding transpires in urban environments,
utilising various obstacles that can be found outdoors, including stairs, rails,
ledges, gaps or flat ground for skaters to showcase their creativity
\cite{skateStyles}.


% \section{Problem Definition}

The problem at hand revolves around the lack of an efficient and objective
method for recognising skateboard tricks in competitions and practice sessions.
Currently, the skateboarding community relies on using subjective methods for
evaluations of these tricks, thus, this can lead to inconsistencies and disputes
in scoring. This lack of objectivity not only disrupts the fairness in
competitions but also inhibits skaters' ability to receive real-time feedback
for skill improvement. 

The integration of Artificial Intelligence (AI) into skateboarding provides
great potential for the standardisation of tricks, real-time feedback and injury
prevention. The development of an AI model that can accurately recognise
skateboard tricks is a tool that the skateboarding community would greatly
benefit from. Consequently, there is a need for the development of an AI model
that can accurately recognise and standardise skateboard tricks, thereby
enhancing the sport's objectivity, fairness and quality.