\chapter{Introduction}
\label{chp:introduction}

Skateboarding dates back to the late 1940s or early 1950s, evolving from a
leisure activity for surfers on flat land to a worldwide phenomenon
\cite{SkateboardingEncyclobeadia}. This worldwide appeal was further amplified
by its inclusion as an official sport in the 2020 Tokyo Olympic Games
\cite{SkatebaordingOlimpics}. This event had a significant impact on the sports
popularity globally, but particularly evident in the United States, as
demonstrated by the spike in the number of U.S. skateboarding participants between
the years 2020 and 2021, illustrated in Figure \ref{fig:skate_graph}.

%Skateboarding is defined by its dynamic tricks, manoeuvrers performed through the manipulation of a skateboard to achieve a desired outcome, manifesting as a popular and "exhilarating “extreme sport”.

This dynamic sport encompasses various disciplines and riding styles, each
offering unique challenges for skateboarders to explore. Two of the most
prominent styles are “vert” and “street.” Vert skateboarding involves
riding on specialised obstacles, namely, half-pipes and ramps, focusing on
aerial manoeuvres. Street skateboarding takes place in urban environments,
utilising various obstacles that can be found outdoors, including stairs, rails,
ledges, gaps or flat ground for skaters to showcase their creativity \cite{skateStyles}.

\begin{figure}[htbp]
	\centering
	\includegraphics[width=0.9 \textwidth]{content/chapters/1_introduction/figures/graph.png}
	\caption{Number of skateboarding participants in the United States from 2010 to 2021 in millions. Reproduced from \cite{skatestatistics}, data sourced from Outdoor Foundation \cite{outdoorfounation}.}
	\label{fig:skate_graph}
\end{figure}

\section{Motivation}
The recent surge in skateboarding's popularity across digital platforms like
YouTube and Instagram have increased the demand for on-screen trick
identification %add a reference
, enabling inexperienced viewers to identify tricks performed by skaters in
videos through a digital overlay that labels each manoeuvre. Traditional methods
require manually labelling each trick in a video, a process that is not only
time-consuming but also error-prone. Automating this process could save time in
this regard, but also offer real-time benefits for live-streamed content, such
as skateboard competitions and events. This automation could provide insightful
footage to viewers, further enhancing their understanding and appreciation of the
sport. An Artificial Intelligence (AI) model capable of accurately classifying
skateboard tricks could significantly improve viewer experience in these
scenarios.

Despite the popularity of computer vision, particularly in video activity
recognition, limited research exists in the domain of skateboard trick
classification. While some initial exploration by Shapiee et al. (2020)
\cite{skatePaper1} and Hanxiao Chen (2023) \cite{SkateboardAIPaper} have
emerged, a gap in this area persists. This significant lack of research in this
area, drives the need for further investigation in this niche field.

% This task poses significant challenges, due to diverse camera angles, lighting
% conditions and complex skateboard movements. However, this research aims to
% overcome these challenges and create a robust and accurate tool that can
% reliably classify skateboard tricks from video.


\section{Hypothesis}
The hypothesis for this research asserts that employing a combination of deep learning strategies and preprocessing techniques can improve the accuracy and robustness of classifying skateboard tricks.

\section{Research Questions}
\begin{itemize}
\item How effectively can deep learning techniques accurately classify skateboard tricks from video data?
\item What is the impact of different video pre-processing techniques on classification accuracy?
\end{itemize}

\section{Aims and Objectives}

\subsection{Aims}
\begin{itemize}
	\item To classify skateboard tricks using images extracted from videos into their respective classes.
	\item To compare the performance of Deep Learning architectures in the context of skateboard trick classification.


\end{itemize}
\subsection{Objectives}
\begin{itemize}
	\item To Implement a set of Deep learning architectures and evaluate them using appropriate metrics.
	\item Employ suitable frame processing techniques
	\item Augment and preprocess the images to determine any performance improvement.
\end{itemize}


\section{Structure}
This study is structured systematically to explore the potential of deep
learning in skateboard trick recognition. Chapter 2 begins by establishing the
necessary background knowledge required for this study, while Chapter 3 provides
an overview of the past research conducted in this domain.

Next, Chapter 4 outlines the approach behind the artefact, with implementation
specifics provided in Chapter 5. Chapter 6, discusses the outcomes and evaluates
them against other studies and finally, Chapter 7 presents the final findings
and identifies limitations that this research encountered.




% \section{Problem Definition}

%The problem at hand revolves around the lack of an efficient and objective
%method for recognising skateboard tricks in competitions and practice sessions.
%Currently, the skateboarding community relies on using subjective methods for
%evaluations of these tricks, thus, this can lead to inconsistencies and
%disputes in scoring. This lack of objectivity not only disrupts the fairness in
%competitions but also inhibits skaters' ability to receive real-time feedback
%for skill improvement.   The integration of Artificial Intelligence (AI) into
%skateboarding provides great potential for the standardisation of tricks,
%real-time feedback and injury prevention. The development of an AI model that
%can accurately recognise skateboard tricks is a tool that the skateboarding
%community would greatly benefit from. Consequently, there is a need for the
%development of an AI model that can accurately recognise and standardise
%skateboard tricks, thereby enhancing the sport's objectivity, fairness and
%quality.

