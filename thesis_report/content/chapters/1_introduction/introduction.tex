\chapter{Introduction}%
\label{chp:introduction}



Skateboarding dates back to the 1940s when handmade skateboards first appeared % date is wrong
\cite{SkateboardingEncyclobeadia}. It has since developed into a worldwide
phenomenon, with its popularity skyrocketing, after gaining recognition as an
official sport in the 2020 Tokyo Olympic Games \cite{SkatebaordingOlimpics}.
Skateboarding is defined by its dynamic tricks, manoeuvrers performed through manipulation of a skateboard to achieve a desired outcome, manifesting as a popular and "exhilarating “extreme sport”. 

This dynamic sport encompasses various disciplines and riding styles, each
offering unique challenges for skateboarders to explore. Two of the most
prominent styles are “vert” and “street.” Vert skateboarding revolves around
riding on specialised obstacles, namely, half-pipes and ramps, emphasising
aerial manoeuvres. Street skateboarding transpires in urban environments,
utilising various obstacles that can be found outdoors, including stairs, rails,
ledges, gaps or flat ground for skaters to showcase their creativity.
\cite{skateStyles}.

The rise of skateboarding video content on platforms like YouTube and Instagram drives demand for the application of on-screen trick identification. This allows inexperienced viewers to identify tricks executed by skaters in videos through a digital overlay displaying the performed trick. Furthermore, automated trick recognition could play a vital role in skateboarding competitions, providing insightful footage to viewers, enhancing the understanding and appreciation of the sport. An Artificial Intelligence (AI) model capable of accurately classifying skateboard tricks could significantly improve viewer experience in these scenarios.

Despite the recent popularity of skateboarding and video activity recognition techniques, limited research exists in
the domain of skateboard trick classification. While some initial studies such as, Shapiee et al. (2020) \cite{skatePaper1} and Hanciao Chen (2023) \cite{SkateboardAIPaper} explored this domain, there remains a gap of research, driving the need for further development in this area.

%talk about limitatiosn of the project - so number of classes

% structure of the paper




% \section{Problem Definition}

%The problem at hand revolves around the lack of an efficient and objective method for recognising skateboard tricks in competitions and practice sessions. Currently, the skateboarding community relies on using subjective methods for evaluations of these tricks, thus, this can lead to inconsistencies and disputes in scoring. This lack of objectivity not only disrupts the fairness in competitions but also inhibits skaters' ability to receive real-time feedback for skill improvement.   The integration of Artificial Intelligence (AI) into skateboarding provides great potential for the standardisation of tricks, real-time feedback and injury prevention. The development of an AI model that can accurately recognise skateboard tricks is a tool that the skateboarding community would greatly benefit from. Consequently, there is a need for the development of an AI model that can accurately recognise and standardise skateboard tricks, thereby enhancing the sport's objectivity, fairness and quality.

\section{Research Question}
How effectively can combining deep learning techniques and algorithms accurately classify skateboard tricks from video data?

