\clearpage
\chapter{Conclusions and Future Work}



\section{Future Work}

{\bf Classification Extension:} As outlined in Section 4.1, this study adopted a
multi-class classification approach, focusing initially on three fundamental
skateboard tricks. While this served as an adequate foundation for exploring the
underlying relationships between these tricks and their application in Deep
Learning, its scope limited its usefulness.

To enhance its applicability, particularly in skateboarding events, the inclusion
of more complex tricks more commonly observed in competitions is crucial.
Additionally, incorporating external entities, such as rails, ledges and gaps
would enhance its real-world usefulness.

\noindent
{\bf Automated Judging Systems:} Currently, skateboarding competitions rely on a panel
of human judges to evaluate skater performance. While judges possess expertise
and experience, this system is heavily subjective. Personal interpretations and
preferences can influence their decisions, raising questions about fairness.

An automated judging system would address these limitations by removing the
element of human bias, facilitating a more fair and consistent scoring system.
However, this will require a more complex classification strategy, as the true
challenge lies in translating all the subjective aspects of style and creativity
into objective criteria for AI scoring.

%analyse sound with video to see if htere is correlation between tricks
%use computer vision techniques and accelermoter techniques together to get better results
\section{Conclusion}