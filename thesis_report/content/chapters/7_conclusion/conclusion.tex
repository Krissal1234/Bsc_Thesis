\clearpage
\chapter{Conclusions and Future Work}

\section{Future Work}
{\bf Classification Extension:} As outlined in Section 4.1, this study adopted a
multi-class classification approach, focusing initially on three fundamental
skateboard tricks. While this served as an adequate foundation for exploring the
underlying relationships between these tricks and their application in Deep
Learning, its scope limited its usefulness in real-world scenarios.

To enhance its applicability, particularly in skateboarding events, the
inclusion of more complex tricks more commonly observed in competitions is
crucial. Additionally, incorporating tricks that interact with external
entities, such as rails, ledges and gaps would enhance its real-world
usefulness.

\vspace{3mm}
\noindent
{\bf Automated Judging Systems:} Currently, skateboarding competitions rely on a
panel of judges to evaluate skater performance live. While judges possess
expertise and understanding, this system is heavily subjective, due to personal
perspectives and preferences influencing their decisions, raising fairness
issues.

An automated judging system would address these limitations by removing the
element of human bias, facilitating a more consistent and fair scoring system.
However, this requires a more complex classification strategy, as the true
challenge lies in translating all the subjective aspects of style and creativity
into an objective score.

\vspace{3mm}
\noindent
{\bf Data limitation:} A major limitation encountered during development was the
insufficient amount of data available to train with. Future work in this area,
would greatly benefit from additional data to expand the model's robustness and
generalisation capabilities. Moreover, if additional data is not accessible, a
possible alternative could involve employing techniques to generate synthetic
data using models such as Generative Adversarial Networks (GANs) or Variational
Autoencoders (VAEs).

\vspace{3mm}
\noindent
{\bf Improvements in evaluation time:} Future work would benefit from
simplifying the expensive video analysis pipeline utilised in this study.
Improvements in processing time will greatly increase its potential in
real-world scenarios, targeting broadcasted skateboard competitions. Possible
approaches include, frame extraction algorithm optimisations, model
simplifications such as using lighter pre-trained feature extractors and
utilising hardware acceleration.
% To address this limitation, future research should focus on optimizing the existing preprocessing steps to reduce their computational demands. Possible approaches might include:

%     Algorithm Optimization: Refining the algorithms used for feature extraction and frame processing to enhance their efficiency.
%     Model Simplification: Exploring lighter, more efficient model architectures that maintain high accuracy while reducing computational load.
%     Hardware Acceleration: Leveraging hardware accelerations, such as GPUs or specialized AI processing units, which can significantly speed up the processing time.
%     Parallel Processing Techniques: Implementing parallel processing techniques to expedite the preprocessing stages, allowing for faster throughput and reduced latency.

%analyse sound with video to see if htere is correlation between tricks
%use computer vision techniques and accelermoter techniques together to get better results
\section{Conclusion}
This study set out to explore the effectiveness of deep learning techniques in
the domain of skateboard trick classification, with a particular focus on
evaluating various preprocessing strategies. The hypothesis stated that the
combination of advanced deep learning models and strategic preprocessing could
enhance the robustness and accuracy of skateboarding trick classification.

Throughout the course of this research, four architectures were implemented:
VGG16-LSTM, VGG16-BiLSTM, ResNet50-LSTM and ResNet50-BiLSTM. These models along
with several preprocessing techniques such as optical flow, PCA and data
augmentation significantly improved classification accuracies and training
efficiencies. This study presents its findings in chapter \ref{6_evaluation},
using relevant performance metrics, thus reaching the objectives listed in
Section \ref{1_objectives}. Notably, the VGG16-BiLSTM model achieved the top accuracy
of 88\%, validating the hypothesis and illustrating the potential of these
techniques in real-world applications such as live broadcasted skateboard
competitions and on-screen trick identification in skateboarding content.

The implications of these findings suggest that the integration of AI solutions
could not only alter the way skateboarding is viewed and analysed but also how
performances across all sports are approached. By automating classification of
sports actions, this technology has the potential to enhance the audience's
understanding and appreciation of the complex maneuvers common to many action
sports.

However, this study faced several limitations, notably its lack of training data
and its applicability in real-time applications, deeming it unsuitable for
real-world scenarios. Despite these challenges, the findings from this study
provide insights into the several techniques and areas for further exploration
that future research could build upon.






% to classify skateboard tricks from video and compare the
% performance of various deep learning architectures in accomplishing this task.
% This research aimed at enhancing live broadcasts and social media content, in
% order to contribute to the possibilities for innovation in the skateboarding
% industry.

% o compare performances of two pre-trained feature extractors
% and sequence models. Evaluations revealed that the selected models were capable
% of matching or exceeding the performance of relevant literature.

% Additionally, this research explored the impact of Principle
% Component Analysis (PCA) and data augmentation on model performance. These
% techniques were employed to assess their potential in improving accuracies and
% efficiencies of the selected model architectures. As detailed in Chapter
% \ref{6_evaluation}, these techniques not only enhanced models' accuracy but also
% reduced training times. presents these experiments with relevant performance
% metrics to demonstrate the benefits of these approaches.