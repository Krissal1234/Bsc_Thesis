\clearpage
\chapter{Conclusions and Future Work}



\section{Future Work}

{\bf Classification Extension:} As outlined in Section 4.1, this study adopted a
multi-class classification approach, focusing initially on three fundamental
skateboard tricks. While this served as an adequate foundation for exploring the
underlying relationships between these tricks and their application in Deep
Learning, its scope limited its usefulness in real-world scenarios.

To enhance its applicability, particularly in skateboarding events, the
inclusion of more complex tricks more commonly observed in competitions is
crucial. Additionally, incorporating tricks that interact with external
entities, such as rails, ledges and gaps would enhance its real-world
usefulness.

\noindent
{\bf Automated Judging Systems:} Currently, skateboarding competitions rely on a panel
of judges to evaluate skater performance live. While judges possess expertise
and experience, this system is heavily subjective. Personal perspectives and
preferences can influence their decisions, raising fairness issues.

An automated judging system would address these limitations by removing the
element of human bias, facilitating a more consistent and fair scoring system.
However, this requires a more complex classification strategy, as the true
challenge lies in translating all the subjective aspects of style and creativity
into objective criteria for AI scoring.

\noindent
{\bf Data limitation:} A major limitation encountered during development was the
insufficient amount of data available to train with. Future work in this area,
would greatly benefit from additional data to expand the model's robustness and
generalisation capabilities. Moreover, if additional data is not accessible, it
could be worth exploring techniques to generate synthetic data using techniques
such as Generative Adversarial Networks (GANs) or Variational Autoencoders (VAEs).




%analyse sound with video to see if htere is correlation between tricks
%use computer vision techniques and accelermoter techniques together to get better results
\section{Conclusion}