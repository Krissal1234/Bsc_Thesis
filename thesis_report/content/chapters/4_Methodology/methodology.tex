\clearpage
\chapter{Methodology}

\section{Dataset }
This study utilised video recordings of skateboarders performing tricks as its
primary data source. To ensure model robustness, the final dataset consisted of
videos across diverse environmental conditions and varying skateboarder skill
levels.

The initial dataset was sourced from the publicly available "SkateboardML"
repository on GitHub \cite{lightningdrop2020skateboardml}, comprising 200 video
clips corresponding to two common tricks: the ollie and the kickflip. This
dataset laid a solid foundation for this research. However, to expand the
dataset's diversity, additional data was obtained through direct communication
with  Hanxiao Chen, the author of the SkateboardAI paper
\cite{SkateboardAIPaper}. This communication yielded a dataset containing 750
videos covering 15 distinct tricks. Given the wide range of tricks included in
this dataset, many were beyond the scope of this study, therefore only a subset
of these videos were carefully selected and incorporated into the final dataset.


\section{Class Establishment}
In the development of a skateboard trick classifier, this study pursued a
multi-class classification strategy, targeting three fundamental skateboard
tricks: ollie, pop shuvit and kickflip. These tricks were selected on the basis
of two primary criteria. Firstly, they are often associated with the first
tricks learnt by beginners, highlighting their role in foundational
skateboarding skills. Secondly, their popularity within the skateboarding
community, often performed in competitions emphasises their relevance, making
them highly relevant for analysing and evaluating competitive performance.


\section{Data Preparation}

\subsection{Labelling techniques}
This study investigated two primary labelling techniques: the folder-based
approach and the text-based approach. In the folder-based method, videos were
categorised into folders named after their corresponding class label offering a
simple organisation method. On the other hand, in the text-based approach, each
video's path and corresponding label were listed on a text file, providing more
flexibility. Given the limited number of classes and manageable dataset size,
this study chose to utilise the folder-based approach, as the extra complexity
from the text-based method wasn't necessary for this project.

 \begin{figure}[h]
 	\centering
 	\begin{minipage}[t]{0.55\textwidth}
 		\includegraphics[width=\textwidth]{content/chapters/4_Methodology/figures/folder-based-label.jpg}
 		\caption{Folder-based labelling.}
 		\label{fig:folder-based-label}
 	\end{minipage}
 	\hfill
 	\begin{minipage}[t]{0.35\textwidth}
 		\includegraphics[width=\textwidth]{content/chapters/4_Methodology/figures/text-based-labelling.jpg}
 		\caption{Text-based labelling}
 		\label{fig:text-based-label}
 	\end{minipage}
 \end{figure}

%Data Acquisition: Describe your dataset:

%How did you collect the data (videos, sensor data, etc.)?
%What types of tricks are included?
%How diverse is the data (skateboarders, environments, angles)?
%How did you label the data (manual labeling, automated tools)? -- two techniques explored, using folders and also using a text file with path and label
\section{Data Augmentation}

\section{ML Architectures}

% VGG-LSTM - because of paper from lit review
% Resnet50-LSTM
% Resnet50-BiLSTM - becuase of skateboard paper and lit review paper

\section{Evaluation Methods}


