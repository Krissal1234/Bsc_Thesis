\clearpage
\chapter{Implementation}

\section{Development Environment}
\subsection{Keras}

\section{Frame Extraction using Optical Flow}
To ensure the selected frames are highly relevant to the task, this study
utilised an optical flow approach. This method prioritises frames with
significant motion, which aligns well with the nature of skateboard tricks. The
core principle behind this study's frame extraction method is the utilisation of
Farneback's algorithm \cite{farneback2003two}. This algorithm

\section{Data Augmentation}
% show parameters for augmentation plus photos of how they were augmented

\section{Callbacks}
\subsection{Early Stopping}
This research employed an early stopping callback to reduce overfitting and save
computational resources. This method monitored the validation loss at every
epoch and halted training if improvement stopped after a predetermined patience
value. This study investigated a pateience of 10 epochs, based on the
observation that the models were unlikely to improve after 10 epochs, with no
validation loss advancements.

\subsection{Model Checkpoint}
This study incroporated model checkpointing in the training process
to save intermediate models after every epoch. This implementation was
configured to monitor the validation loss and only save the the model when it
showed an improvement, allowing a seamless resumption of training in case of
interruption.

\section{}