
% given that this is a video recognition task - we require the use of extracting frames

% Talk about the differnet techniques of frame extractin due to the short vide
% uniform sampling
% think about using optical flow to extract frames that show motion
% clips -- figuring out how many extra frames required, then duplicating at the
% end, -- also tried duplicating randomly in the middle, which did not produce
% good results

%Yes, you should definitely mention the two frame extraction methods you explored in the methodology section. Here's how to approach it and what you would include in both methodology and implementation:
%
%Methodology
%
%Introduce the need for frame extraction: Briefly reiterate the importance of frame extraction for video recognition tasks, especially when dealing with activity recognition.
%Describe the two techniques:
%Optical Flow: Explain the concept of using optical flow to calculate motion, and how you utilized it to assign weights to frames, prioritizing those with higher motion values for selection.
%Uniform Sampling: Describe the process of extracting frames at regular intervals and how you applied it.
%Evaluation Criteria: Mention the criteria used to compare the techniques (e.g., ability to capture key actions, reduce redundancy, maintain temporal consistency).
%Justification of Choice: Briefly state the method that performed better based on your evaluation criteria and explain the reason for its selection.
%
%Implementation
%
%Chosen Method: State the chosen technique (e.g., optical flow-based extraction).
%Software/Libraries: Mention the specific tools used to implement this (e.g., OpenCV).
%Specific Parameters: Describe any key parameters or thresholds you configured for the optical flow calculation and frame selection.
%Integration: Briefly explain how the extracted frames were used in your activity recognition model (e.g., input into a convolutional neural network).
%
%Discussion of Results
%
%Whether you should include a detailed discussion of results for both techniques depends on the scope of your thesis and how central frame extraction is to your core research findings. Here are some options:
%
%Core Focus: If frame extraction is a crucial element, include a comparative analysis of the results obtained from both methods and how they influenced your final choice.
%Secondary Focus: If frame extraction is less central, briefly summarize the outcome in the methodology section, stating why the chosen method yielded better results for your task.
%Results Section: You can dedicate a subsection in your separate results section to discuss the performance of both frame extraction techniques in more detail.

