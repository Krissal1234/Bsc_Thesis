\chapter*{Abstract}
\addcontentsline{toc}{chapter}{Abstract}

This study presents an AI-based approach to skateboard trick recognition, aimed
at automating the identification of tricks from videos. The underlying
motivation for this research is to enhance viewer engagement in skateboarding
content on digital platforms and live-streamed events by providing real-time
trick identification.

This research explores the effectiveness of preprocessing techniques and Deep
Learning strategies to identify which approaches can improve accuracy and
robustness in classifying skateboard tricks. It explored preprocessing
techniques such as optical flow for frame extraction, data augmentation for
dataset enhancement and PCA for dimensionality reduction. Subsequently, the
study investigates four deep learning architectures: VGG16-LSTM, VGG16-BiLSTM,
ResNet50-LSTM, and ResNet50-BiLSTM, leveraging pre-trained models: VGG16 and
ResNet50 for feature extraction followed by sequence models using LSTM and
BiLSTM networks.

The results indicate that the VGG16-BiLSTM model achieved the highest accuracy
of 88\%, outperforming other architectures. In addition to these findings, a
number of experiments were conducted, including the application of data
augmentation, the choice of optimiser, the use of PCA for dimensionality
reduction and the models' real-world applicability. Data augmentation resulted
in an improvement of 7.75\% and the use of PCA for dimensionality reduction was
observed to drastically reduce training costs and improve accuracy.

The findings from this study suggest that Deep Learning models, can effectively
classify skateboard tricks from video data, offering potential applications in
enhancing viewer experience during skateboard competitions and content consumption.